\documentclass[../introduction.tex]{subfiles}
\begin{document}
\section{But du projet}
Ce projet a pour objectif de créer une expérience immersive où deux joueurs s'affrontent dans le célèbre jeu de société "Bataille navale".
Chacun des joueurs dispose d'un plateau de jeu unique, dissimulé à son adversaire, sur lequel il place stratégiquement ses navires. 
Le déroulement du jeu se fait tour à tour, avec les joueurs tentant de localiser et de couler les navires adverses en lançant des attaques sur des coordonnées spécifiques.
Dans le jeu originel, lorsqu'une attaque atteint un navire, la case correspondante est signalée par un point rouge ; autrement, elle est marquée d'un point blanc.
La victoire est remportée par le joueur qui parvient à couler tous les navires adverses.\\

Deux modes de jeu sont disponibles : le mode classique et le mode commandant.\\

Mode classique:\\
Les flottes des deux joueurs se composent de cinq navires de tailles variées : 2x1, 3x1, 3x1, 4x1, et 5x1. 
Ces navires peuvent être orientés de 90 degrés, mais leur placement en diagonale n'est pas autorisé. 
De plus, deux navires ne peuvent pas être adjacents l'un à l'autre. 
Chaque tour permet à un joueur de tirer un coup, mais tant que le tir touche un navire, le joueur peut continuer à tirer.\\

Mode commandant:\\
Dans le mode commandant, différentes factions sont introduites, chacune avec ses propres navires et compétences distinctes. 
Contrairement au mode classique, dans ce mode, deux navires peuvent être positionnés côte à côte. 
Au début de son tour, chaque joueur reçoit des points d'énergie. 
Chaque action, comme le tir, consomme des points d'énergie. 
De nouvelles actions sont également disponibles, telles que l'utilisation d'un sonar révélant une zone de 3x3, 
le déclenchement d'un bombardement aérien ciblant une zone de 4x1 et bien d'autres.
Si un joueur n'utilise pas tous ses points d'énergie au cours d'un tour, il peut les conserver pour le tour suivant, offrant ainsi une flexibilité tactique pour planifier des actions stratégiques plus complexes.\\

Un mode "extra" supporter est également présent. Dans ce mode, l'utilisateur observe une partie jouée par un ami.\\

En dehors d'une partie, un joueur a la possibilité de discuter avec ses amis et de gérer sa liste d'amis.
\end{document}
